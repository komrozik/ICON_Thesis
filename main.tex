\documentclass[11pt]{article}
\renewcommand{\baselinestretch}{1.5}
% Language setting
% Replace `english' with e.g. `spanish' to change the document language
\usepackage[english]{babel}

% Set page size and margins
% Replace `letterpaper' with `a4paper' for UK/EU standard size
\usepackage[letterpaper,top=2cm,bottom=2cm,left=3cm,right=3cm,marginparwidth=1.75cm]{geometry}

% Useful packages
\usepackage{amsmath}
\usepackage{graphicx}
\usepackage[colorlinks=true, allcolors=blue]{hyperref}
\usepackage[version=4]{mhchem}

%\usepackage[backend=biber,style=science]{biblatex}
%\addbibresource{~/Library/texmf/bibtex/bib/Zotero.bib}

\usepackage{natbib}
\bibliographystyle{plainnat}


\title{Master Thesis FIRST IDEAS}
\author{Konstantin Mrozik}

\begin{document}

\begin{titlepage}
    \begin{center}
        \vspace*{1cm}
            
        \Huge
        \textbf{SPEC Report}
            
        \vspace{0.5cm}
        \LARGE
        Ocean acidification and carbonate saturation in the coastal ocean
            
        \vspace{1.5cm}
            
        \textbf{Konstantin Mrozik}
        
        \vspace{0.8cm}
        Supervisors:\\
        Tatiana Ilyina\\
        Moritz Mathis
            
        \vfill
            
        Report for the SPECIALIZATION Course\\
        Ocean and Climate Physics Master
            
        \vspace{0.8cm}
            
        \includegraphics[width=0.4\textwidth]{uhh-logo-insert}
            
        \Large
        Department Name\\
        University of Hamburg\\
        Germany\\
        \today
            
    \end{center}
\end{titlepage}

\tableofcontents   

\newpage


\begin{abstract}
With higher atmospheric $\ce{CO2}$ levels, the amount of dissolved inorganic carbon in the world's oceans also steadily increases. Due to this increase in oceanic carbon, the ocean's pH has declined measurably since the Industrial Revolution. With this thesis approach, I want to investigate the anthropogenic impact on these coastal systems and especially regarding biogeochemical processes.

The ocean's pH has a major impact on calcifying species in the water and is therefore important for the entire food chain. The coastal ocean is a hotspot for biogeochemical processes due to high biological activity and the chemical influx from the land. Even though ocean acidification has been studied on a large scale in Earth system models, the coastal area is usually poorly represented in models, especially in terms of its biogeochemistry.

With this project, I aim to use a new model representation of the coastal oceans and specifically their biogeochemistry to investigate the Carbonate Saturation State ($\Omega$). ICON-Coast offers higher coastal resolution and improved biogeochemical parametrisations. The Carbonate saturation state links increases in dissolved inorganic carbon to key biogeochemical processes.
A global view is needed to identify vulnerable regions before zooming into case studies; therefore, by first investigating on the global scale, I aim to distinguish particularly vulnerable areas to take a deeper look into the regional influences. I will specifically consider the different parameters influencing the Carbonate Saturation State and its variability and extreme events of the past.
\end{abstract}
 

\section{Introduction}



The global Ocean has taken up $29\%$ of total emissions from the past century and therefore plays a major role for the earth climate and future (\citet{Friedlingstein2025a}).
Additionally, to the ocean the land took up $21\%$  of the total emissions in the past decade (\citet{Friedlingstein2025a}).
In between the ocean and the land sits the coastal and shelf seas, which have a specific role connecting both sinks. 
Coastal oceans provide a variety of ecosystem services, acting as breeding grounds, barriers for waves and flooding and the absorption of anthropogenic CO2 is one of the most important ecosystem services provided by the coastal ocean (Cooley et al. 2009).
While the shallow shelf and marginal seas only cover about $7\%$ of the total ocean area, they are responsible for an uptake of about $0.19$ to $0.25\; \text{Pg}$ Carbon per year, (\citet{roobaertSpatiotemporalDynamicsSources2019}, \citet{laruelleRegionalizedGlobalBudget2014}) having a flux density $~40\%$ more intense than that of the open ocean (\citet{Dai2022}).

This increased uptake invites the question of why this is the case for the coastal ocean and why it plays this special role.
The answer to these progresses can be partly inferred from the biochemical and physical processes governing this region. 
The coastal ocean is influenced by riverine input, upwelling of deep water as well as coastal degradation.
All the aforementioned processes adding nutrients to this shallow water in the epipelagic zone.
The combination of high nutrient load and sunlight exposure results in unusual high primary production and carbon uptake from biology ().
Due to the shallow water depth in the coastal ocean and the connection to the land interface, storm tides and high wave action not only affect the direct beach/cliff area but also lead to an increased resuspension of sediments in the shelf seas (). 

As discussed by \citet{gruberCarbonCoastalInterface2015} the current measurements of coastal carbon not yet answer the question how human activtiy will 



\subsection{Motivation: Coastal Oceans and the Global Carbon Cycle}

About $0.19-0.30$ GtC yr$^{-1}$ of \ce{CO2} get absorbed by the coastal ocean according to recent studies \citet{laruelleRegionalizedGlobalBudget2014, bauerChangingCarbonCycle2013, roobaertNovelSeaSurface2024, regnierLandtooceanLoopsGlobal2022, Dai2022}

\begin{enumerate}
  \item Coastal Oceans as the interface between Land and Ocean
  \item high biological activity, high primary production
\end{enumerate}

\subsection{Ocean Acidification and Carbonate Chemistry}

\begin{enumerate}
  \item What is Ocean Acidification 
  \item How is OA linked to Climate Change 
  \item How does OA interact with the Carbon Cycle
\end{enumerate}



\subsection{Carbonate Saturation State ($\Omega$) and Coastal Vulnerability}

\begin{enumerate}
  \item Whats Omega
  \item how is omega linked to bilogy and Chemistry
  \item whay is the coast specifically vulnerable 
\end{enumerate}


\subsection{Research Gap and Objectives}


% 
% \subsubsection{Physics} 
% \label{section:physics}
% % You say polar waters are “strongly saturated with CO₂” — it’s true they hold high DIC, but “saturated” is misleading in the carbonate chemistry context.
% % You didn’t mention why physical processes matter for carbon uptake (ventilation, residence time, stratification) — that’s the key connection to the carbon cycle.
% % Heat/freshwater fluxes: the explanation is vague (e.g. “western boundary currents move warm water poleward” is true, but how does that connect to air–sea heat exchange and CO₂ solubility?).
% 
% The physical ocean carbon cycle is the part driven by physical processes. Due to advection, diffusion and stirring, the biochemical tracers are moved and spread in the global oceans.
% The Atlantic Meridional Overturning Circulation (AMOC) is a thermohaline circulation and one of the biggest planet-scale movements in the ocean. With the AMOC, Carbon, Nutrients and Heat get transported all over the globe. In the polar regions, typically, the Oceans' Deep water is formed. High saline and cold water, strongly saturated with Carbon dioxide, sinks in the ocean basin and flows equatorward. With this cold water movement, a balancing flow of warm tropical surface waters towards the pole is established. This balancing flow transports Plankton and Heat Poleward. With this transport of Heat, Nutrients and Carbon, the AMOC affects the ability of the surface ocean to take up additional CO2 and also impacts the long-term storage capabilities of the ocean.
% 
% % Upwelling/downwelling section: the description of Ekman transport and cyclone-driven divergence is garbled. Terms like “mass conversation” (should be conservation) are distracting errors, and the reasoning is not rigorous enough for a methods/theory section
% Ocean upwelling is the process of upward movement of deep water due to wind stress on surface waters. Ekman currents typically develop in coastal regions near land or when there are circular winds, like in a Cyclone. When the wind blows equatorward and parallel to the coastline, it induces a Coriolis-driven flow in the water perpendicular to the direction of the wind blowing. If this current moves surface waters away from the coastline, it induces deeper water upwelling due to mass conservation. If the surface current travels away from the coast, a decrease in surface water is induced. 
% In Cyclones, it follows a similar reasoning: if the surface current is induced away from the eye of the cyclone, upwelling is induced; otherwise, it will induce local downwelling.
% 
% % The air–sea flux section (momentum, heat, freshwater) reads like a loose afterthought, not connected back to the carbon cycle.
% These Ocean circulation patterns are mainly driven by the Air-Sea interaction and exchanges of Momentum, Heat, and Freshwater. Momentum transfer follows the general atmospheric circulation patterns and local winds. The heat transfer has the pattern of maximal heating in the tropics and cooling in the polar regions; near the western coastal area, there is usually higher cooling due to the western boundary currents moving warm water poleward. Freshwater exchange follows a complex pattern, mainly depending on regions of air rising in the atmosphere, typically near the tropics or mid-latitudes. Evaporation occurs in regions of warmer waters.
% 
% \subsubsection{Biology}
% Biology: biological pump basics (organic export, remineralization, role in CO2 uptake).
% 
% The atmospheric carbon dioxide enters the ocean through interactions at the sea surface or via coastal and riverine degradation fluxes \ref{section:physics}.
% Upon contact with the ocean water, the carbon dioxide undergoes a chemical reaction, resulting in the formation of carbonate and bicarbonate ions \ref{section:chemistry}.
% The next stage of the process involves ocean biology, which is responsible for the absorption of carbon within the biomass. 
% The resultant carbon, having become fixed, is exported into the deep ocean and subsequently redistributed through various processes.
% In this section, the focus will be on an analysis of the processes involved.
% 
% \begin{figure}
%     \centering
%     \includegraphics[width=0.75\linewidth]{figures/PP_dist_1998-2013.jpg}
%     \caption{The distribution of the mean net annual Primary Productivity throughout the global ocean based on satellite ocean colour data from 1998 through 2013 (LME boundaries are shown as black polygons) \citet{oreillyOReilly2017StatusTrends2017}}
%     \label{fig:PP_dist}
% \end{figure}
% 
% The primary biological activity in the ocean is that of phytoplankton and zooplankton. 
% The dissolved carbon is absorbed by phytoplankton in the primary production of the ecosystem. 
% The transformation of dissolved carbon into biomass is achieved through the process of photosynthesis, in which plankton play a pivotal role.
% During this process, they absorb carbon and nutrients (P, N, Fe) and concurrently release oxygen.
% The subsequent interaction between the phytoplankton and the zooplankton is of particular interest. The former is either grazed by the latter or dies naturally.
% Zooplankton serve as a source of nutrition for higher trophic levels in the ecosystem. They also export organic matter, either in the form of faecal pellets or as dead detritus, when they die.
% In addition to the fundamental biological cycle, the exported organic matter (detritus and faecal pellets) will be subject to the influence of bacteria as it descends and sinks into the deep ocean.
% The process of bacterial action involves the transformation of organic matter, resulting in the remineralisation of nutrients and the uptake of oxygen from the surrounding water.
% The organic matter, which has not undergone remineralisation, reaches the ocean floor, where the carbon is stored in the ocean's sediment.
% 
% In the majority of cases, the rates of photosynthesis and primary production are constrained by the availability of light and nutrients. Consequently, the greatest productivity is typically observed in the euphotic zone.
% Nutrients are typically supplied by upwelling \ref{section:physics}, vertical mixing, or riverine fluxes.
% The primary biolimiting nutrients are iron, silicate, nitrate, and phosphorus.
% Redfield Ratio
% 
% In the oceanic environment, a diverse array of phytoplankton species is present, each playing a distinct role in the carbon cycle.
% Diatoms are unicellular phytoplankton that form silica shells; this makes them significant in terms of the uptake and export of silicate to the deep ocean.
% 
% Coccolithophores are defined as autotrophic phytoplankton that also build cells, but in this case out of calcium carbonate (\ce{CaCO3}). Coccolithophores play a pivotal role in the binding of carbon, both within their biomass and in the composition of their shells. The coccolithophore shell is composed of calcium carbonate, a chemical compound that is found in the skeletons of many organisms. This chemical composition renders the species particularly sensitive to alterations in the carbonate saturation of the ocean's water. 
% In conditions of undersaturation, these organisms are unable to construct protective shells and may even experience adverse effects from the dissolution of their own shells, which can ultimately result in the demise of the species.
% 
% In addition to these shell-building plankton species, the ocean also hosts a significant number of heterotrophs, i.e. non-shell-building but phytosynthetic organisms.
% This comprehensive approach encompasses the three primary classifications of phytoplankton and encompasses all species depicted within the model.
% 
% \vspace{1cm}
%  
% Part two: Export and Remineralisation
% Maybe not that important for my topic,
% keep it short
% 
% While the primary production creates organic compounds out of the dissolved carbon in the ocean with photosynthesis, another big part of the biological carbon cycle is the export production of bacteria in the aphotic zone.
% Between 4-15\% of the primary production is exported to the deeper ocean and sinks through the ocean while being subject to bacterial activity.
% Most of the Organic matter is converted back to its inorganic constituents through remineralisation.
% Particulate matter sinking through gravity, as well as dissolved organic matter, moved with the circulation to the lower ocean, is affected by this process.
% In the Remineralisation process, bacteria convert organic matter and an oxidant to the inorganic Nutrients and additionally carbon dioxide and water. 
% The oxidant can either be dissolved oxygen in aerobic waters (respiration) or Nitrate (\ce{NO3}) in anaerobic water, which is known as denitrification. Then, in anaerobic conditions where nitrate is unavailable, Sulfate (\ce{SO4}) can be used, leading to sulfate reduction.
% 
% Due to this interaction with the nutrients and the liberation of the inorganic compounds, remineralisation plays a big role in the carbon and nutrient cycles.


% | Area          | Feedback  |
% | ------------- | -------------------------------------------------------------------------------------- |
% | **Structure** | Break into 3 subparagraphs (uptake, export/remineralisation, nutrient limits).         |
% | **Relevance** | Add 1–2 sentences linking biology → CO₂ balance → Omega / acidification.                   |
% | **Precision** | Remove physical/chemical overlaps; tighten terms (fixation, export, remineralisation). |
% | **Style**     | Simplify phrasing, avoid filler and repetition.                                        |
% | **Details**   | Integrate Redfield Ratio properly; clarify sediment burial fraction.                   |

% \subsubsection{Chemistry}
% \label{section:chemistry}
% More than 99\% of the DIC is stored in the form of bicarbonate and carbonate ions \citet{Williams2011}.
% 
% carbonate system basics (pH, DIC, TA, omega).
% In itself the Seawater is a special case of a system compared to the freshwater on land.
% Seawater has salts in it which makes the largest difference chemically and has major influence on the carbon cycle.
% Due to the dissolved salts the seawater is denser than regular freshwater, this density difference is together with density differences due to temperature the driver of the AMOC.
% 

% \subsection{Redfield Ratio}
% Fixed molar ratio of main chemical elements in living Cells.
% If we consider the Photosynthesis in living Phytoplankton, we get:
% \begin{equation}
%    \ce{106CO2 + 16 NO3- + H2PO4- + 122 H2O ( + Photons) -> C106H246O110N16P + 138 O2}
% \end{equation}
% From this endproduct, we can see the relation of $ C:N:P:\ce{O2} = 106:16:1:-138$, which is the so-called Redfield Ratio, which was measured by Redfield for C:N:P to estimate \ce{O2}.
% Redfield et al measured it for Zooplankton and Phytoplankton separately, but the now known Redfield Ratio is the mean of both measurements \citet{redfield1963}.
% \subsection{Liebigs Law}
% I don't remember what that is.
% 
% \subsection{Michaelis and Menten (Monod) Kinematics}
% Predator Prey Dynamics
% \subsection{Nutrient Recycling}
% 
% \begin{align*}
%     \frac{dN}{dt} &= -r_{max} \frac{NP}{k_N + N} +l_{PN} N + l_{DN} D\\
%     \frac{dP}{dt} &= r_{max} \frac{NP}{k_N + N} -l_{PN} P - l_{PD} P\\
%     \frac{dD}{dt} &= l_{PD} P - l_{DN} D\\
% \end{align*}
% 
% \subsection{Revelle factor}
% The Revelle factor (buffer factor) is the ratio of the change of Carbon dioxide to the change of dissolved inorganic carbon.
%   
% \begin{equation}
%   \text{Revelle factor} = \frac{\frac{\Delta \left[\ce{CO2}\right]}{\left[\ce{CO2}\right]}}{\frac{\Delta \left[DIC\right]}{\left[DIC\right]}}
% \end{equation}
% The Revelle factor is called a buffer factor since it defines how the ocean buffers the actual uptake of \ce{CO2} by converting it into Bicarbonates and Carbonate ions resulting in a smaller change of dissolved carbon dioxide compared to the actual uptaken carbon dioxide.
% The lower value the revelle factor has the larger is the buffering capacity of the ocean and more \ce{CO2} can be taken up from the atmosphere.


\section{Background}

\subsection{The Marine Carbonate System (DIC,TA,pH,$\Omega$)}

\subsection{Processes Influencing Carbonate Chemistry in Coastal Oceans}

\subsection{Implications for Calcification and Ecosystems}



\section{Methods}

\subsection{Model Description}

The ICOsahedral Non-hydrostatic Earth System Model (ICON-ESM) is a Earth System Model developed by the Max Planck Institute for Meteorology (MPI-M) for climate predictions (\citet{Jungclaus2022a}).
The Model is based on the ICON Framework from the German Weather Service (Deutscher Wetterdienst, DWD), the MPI-M, the Karlsruhe Institute of Technology (KIT) and other partner institutions.
ICON-ESM consists of different components for the different parts of the Earth System, combining the ocean ICON-O, the atmosphere ICON-A and the Land ICON-Land in one coupled Model.
Biogeochemical aspects of the ocean in the ICON-ESM are calculated by the ocean biogeochemistry module Hamburg Ocean Carbon Cycle (HAMOCC; \citet{ilyinaGlobalOceanBiogeochemistry2013}).

Apart from the original model configuration, multiple variations have been developed focusing on different aspects of the earth system.
At the MPI-M and the Helmholtz Zentrum Hereon a new version of ICON-O has been developed, called ICON-Coast (\citet{mathisSeamlessIntegrationCoastal2022}).
% ICON-Coast is a advancement of the ICON-Ocean model used in the newest iteration of the Earth System Model from the Max Planck Institute for Meteorology.
With ICON-Coast several processes specifically important in coastal systems are implemented in ICON-O and it also introduces a telescopic grid with the possibility to have different mesh sizes in the shelf seas than in the open ocean (Figure \ref{fig:ICON_coast_grid}).

\subsubsection{ICON-O Description}

ICON-O is a global ocean model based on the ICON Framework. The Oceans dynamics in ICON-O are represented by conservative discretization of the ocean primitive equations on a unstructured triangular grid. 
Here, a quick description of ICON-O is given focusing on the processes most relevant for this study and to put in context the advancements of ICON-Coast \citet{mathisSeamlessIntegrationCoastal2022}.
The variables in the grid are in a C-type staggering, locating the scalar variables in the center of each cell and normal components of flux variables on the cell boundaries (\citet{kornFormulationUnstructuredGrid2017} \citet{logemannGlobalTideSimulations2021})
The biogeochemical component of ICON-O is HAMOCC with its CMIP6 configuration. 
The marine biology is represented in a extended NPZD-type model (\textcolor{red}{REFERENCE TO THEORY?}) (\citet{Six1996}) and carbon and nutrient sequestration by phytoplankton is controlled by light availability, water temperature and macro nutrient limitation of phosphate and nitrate assuming Redfield stoichiometry (\textcolor{red}{REFERENCE TO THEORY?}).
% Bacterial decomposition, denitrification and sulfate reduction are distinguished for oxic, sub- and anoxic conditions.
% The Nitrogen Cycle has a prognostic representation of N-fixation at the sea surface by cyanobacteria.14:26
Additionally a 3-dimensional sediment model is included in ICON-O to account for deposition and dissolution of particulate organic matter at the sea floor and benthic pore water exchange.

\subsubsection{ICON-Coast advancements}

% Restructure this later into Physical (grid,tides), Sediment, Biogeochemistry/carbon export, Riverine fluxes

In ICON-Coast several processes which are relevant for the coastal ocean are added as well as a telescopic grid which introduces a higher horizontal resolution depending on the water depth. With this improvements the goal is to catch specific coastal phenomenon and describe the important part of the coastal ocean better in the global models.
Among the changes are the following:
Tidal currents are implemented to include specific tidal dynamics including partial tide interactions but disregarding effects of loading and self-attraction. (\textcolor{red}{research what all this means})
Sediment resuspension is included as a dependence on mean sediment density and grain size in each bottom cell. The resuspension also depends on the erosion depth which is deducted from the bottom current velocities of the model. In ICON-Coast the resuspension is not only including the erosion of solid constituents but also the mixing of pore water released by the bottom erosion.
The export dynamics of biogenically bound carbon are represented by a scheme following Maerz et al (2020) and explicitly accounts for the influence of size, microstructure, heterogeneous composition, density and porosity of the aggregates on their settling velocities and exposure to transformation processes.
While marine aggregates are explicitly represented, also a temperature dependence for remineralization and dissolution processes is introduced. To represent the changes and the effect of moving deposited matter by resuspension to different areas a temperature dependent remineralization is included.
Also in the upper water column this approach is introduced for the organic matter.
River mouths are introduced as point sources of riverine matter fluxes and freshwater.
From rivers also terrestrial dissolved organic matter is introduced in the ocean. Since it is more refractory and has a higher carbon to nutrient ratio it has a different C:P ratio in the model and the mineralization of this new tracer is adjusted as well. 

With these changes the coastal carbon cycle is represented in higher detail compared to ICON-O and it is expected that these changes will influence the coastal oceans chemistry, affecting acidification rates and the carbonate saturation.
The processes introduced are interlinked to all other processes happening in the open and coastal ocean and therefore it is important to work with a model as holistic as possible to capture possible feedbacks between physical, chemical and biological processes.

\begin{figure}[ht]
    \centering
    \includegraphics[width=0.75\linewidth]{figures/ICON_coast_grid.png}
    \caption{Non-uniform grid configuration used in ICON-Coast simulations. In this example, the highest horizontal resolution in the coastal ocean and continental margins is 10 km. The degree of grid refinement is locally dependent on the distance to the coastline, the water depth and the slope of the bottom topography (\citet{logemannGlobalTideSimulations2021} \citet{mathisSeamlessIntegrationCoastal2022})}
    \label{fig:ICON_coast_grid}
\end{figure}

% \begin{figure}[ht]
%     \centering
%     \includegraphics[width=0.75\linewidth]{figures/mean_biomes.png}
%     \caption{Mean biome map created from mean climatologies of maxMLD, SST, summer Chl a, and maximum ice fraction. Dark blue: ice biome (ICE); cyan: subpolar seasonally stratified biome (SPSS); green: subtropical seasonally stratified biome (STSS); yellow: subtropical permanently stratified biome (STPS); orange: equatorial biome (EQU). White indicates ocean areas that do not fit the criteria for any biome and are excluded from further analysis. \citet{fayGlobalOpenoceanBiomes2014}}
%     \label{fig:mean_biomes}
% \end{figure}


\subsection{Model Output and variables}

\textcolor{red}{Draft- will be refined}

In this project, we will use a coupled run of the ICON-Coast model with a grid size ranging from $80\; \text{km}$ in the open ocean to $10\; \text{km}$ in the coastal land-ocean interface.
The model run which will be analyzed in this project is a run from 1900 to 2010 with monthly output, allowing to capture the development of variables through the 20th century.


Through HAMOCC multiple biogeochemical variables are available for analysis.
\newline
Dissolved Inorganic Carbon (DIC) is output directly by the model and refers to the total amount of carbon dioxide $\ce{CO_2}$ (plus carbonic acid $\ce{H_2CO_3}$) , bicarbonate ions $\ce{HCO_3^-}$ and carbonate ions $\ce{CO_3^2-}$ in seawater. 
DIC is strongly connected to the acidification of seawater since the pH is affected by the ratio the three constituents to each other.
\begin{equation*}
  \text{DIC} = \left[\ce{CO_2}\right] + \left[\ce{HCO_3^-}\right] + \left[\ce{CO_3^2-}\right]
\end{equation*}

Total Alkalinity (TA) is the capacity of water to resist acidification.
It is related to the charge balance in sea water and is calculated as follows:
\begin{equation*}
  \text{A}_\text{T} = \left[\ce{HCO_3^-}\right] + \left[\ce{CO_3^2-}\right] + \left[\ce{B(OH)_4^-}\right] + \left[\ce{H^+}\right] + \text{minor compounds}
\end{equation*}
Alkalinity is affected by the remineralization of organic matter by micro algea, precipitation and by dissolution of calcium carbonate \citet{wolf-gladrowTotalAlkalinityExplicit2007} which links it directly to ocean acidification and the carbonate saturation state.

Temperature (T) and Salinity (S) are  variables extracted from the physical part of ICON-O, as physical variables they have a influence on biological activity and chemical properties and therefore affect acidification indirectly.

Partial pressure of carbon dioxide ($\ce{pCO_2}$) measures the contribution of carbon dioxide to the total gas pressure and therefore give insights to the air-sea carbon flux.
Due to the formation of carbonic acid and further chemical reactions, $\ce{pCO_2}$ deviates slightly from the expected values from the classical Henry's Law.

The carbonate saturation state ($\Omega$) is not directly calculated by the model but can be diagnosed by using carbonate ion concentration and the solubility product $K_\text{Sp}^*$.
\begin{align*}
  \Omega &= \frac{\left[\ce{Ca^2+}\right]_\text{SW} \times \left[ \ce{CO_3^-} \right]_\text{SW}}{K_\text{Sp}^*} \\
  K_\text{Sp}^* &= K_\text{Sp}^*(T,S,p)
\end{align*}
Where the solubility product $K_\text{Sp}^*$ depends on the physical properties of the surrounding water and the concentration of Calcium ions is assumed constant in the ocean.

\subsection{Coastal vs Open-Ocean Separation}


\subsection{Analysis Strategy}

\subsubsection{Trend and Variability Analysis}

\subsubsection{Conceptual Driver decomposition}

For the decomposition of how the signal is influenced by different drivers I will use the approach from Takahashi et al. (\citet{takahashiSeasonalVariationCO21993}).
Takahashi originally decomposed the change of partial pressure of \ce{CO2} into the different changes driven by temperature, salinity, alkalinity and dissolved inorganic carbon.
\begin{equation}
  \Delta p \ce{CO2} \equiv p\ce{CO2} \left[ \gamma_T \Delta T + \gamma_S \frac{\Delta S}{S} + \gamma_{Alk} \frac{\Delta Alk}{Alk} + \gamma_{DIC} \frac{\Delta DIC}{DIC} \right] + \epsilon
\end{equation}
For his approach Takahashi used different Revelle (DIC) and Buffer (general) factors to get the right factors for the partial derivatives.


Multivariate linear Regression

\begin{equation}
  y_i = \beta + \beta_1 x_{i1} + \beta_2 x_{i2} + ... + \beta_p x_{ip} + \epsilon
\end{equation}
  with $i = n$ Observations.

There are multiple assumptions following from the multivariate linear Regression approach.
\begin{itemize}
  \item There is a linear relationship between the dependent variable and the independent variables 
  \item The independent variables are uncorrelated 
  \item The samples of the dependent variable are chosen randomly
  \item Residuals are normally distributed with a mean of zero and a variance of $\sigma$
\end{itemize}




\newpage

% \bibliographystyle{apalike}
\bibliography{Zotero}
%\bibliography{~/Library/texmf/bibtex/bib/Zotero}
\newpage

%\printbibliography

\end{document}
