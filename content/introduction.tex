\section{Introduction}



The global Ocean has taken up $29\%$ of total emissions from the past century and therefore plays a major role for the earth climate and future (\citet{Friedlingstein2025a}).
Additionally, to the ocean the land took up $21\%$  of the total emissions in the past decade (\citet{Friedlingstein2025a}).
In between the ocean and the land sits the coastal and shelf seas, which have a specific role connecting both sinks. 
Coastal oceans provide a variety of ecosystem services, acting as breeding grounds, barriers for waves and flooding and the absorption of anthropogenic CO2 is one of the most important ecosystem services provided by the coastal ocean (Cooley et al. 2009).
While the shallow shelf and marginal seas only cover about $7\%$ of the total ocean area, they are responsible for an uptake of about $0.19$ to $0.25\; \text{Pg}$ Carbon per year, (\citet{roobaertSpatiotemporalDynamicsSources2019}, \citet{laruelleRegionalizedGlobalBudget2014}) having a flux density $~40\%$ more intense than that of the open ocean (\citet{Dai2022}).

This increased uptake invites the question of why this is the case for the coastal ocean and why it plays this special role.
The answer to these progresses can be partly inferred from the biochemical and physical processes governing this region. 
The coastal ocean is influenced by riverine input, upwelling of deep water as well as coastal degradation.
All the aforementioned processes adding nutrients to this shallow water in the epipelagic zone.
The combination of high nutrient load and sunlight exposure results in unusual high primary production and carbon uptake from biology ().
Due to the shallow water depth in the coastal ocean and the connection to the land interface, storm tides and high wave action not only affect the direct beach/cliff area but also lead to an increased resuspension of sediments in the shelf seas (). 

As discussed by \citet{gruberCarbonCoastalInterface2015} the current measurements of coastal carbon not yet answer the question how human activtiy will 



\subsection{Motivation: Coastal Oceans and the Global Carbon Cycle}

About $0.19-0.30$ GtC yr$^{-1}$ of \ce{CO2} get absorbed by the coastal ocean according to recent studies \citet{laruelleRegionalizedGlobalBudget2014, bauerChangingCarbonCycle2013, roobaertNovelSeaSurface2024, regnierLandtooceanLoopsGlobal2022, Dai2022}

\begin{enumerate}
  \item Coastal Oceans as the interface between Land and Ocean
  \item high biological activity, high primary production
\end{enumerate}

\subsection{Ocean Acidification and Carbonate Chemistry}

\begin{enumerate}
  \item What is Ocean Acidification 
  \item How is OA linked to Climate Change 
  \item How does OA interact with the Carbon Cycle
\end{enumerate}



\subsection{Carbonate Saturation State ($\Omega$) and Coastal Vulnerability}

\begin{enumerate}
  \item Whats Omega
  \item how is omega linked to bilogy and Chemistry
  \item whay is the coast specifically vulnerable 
\end{enumerate}


\subsection{Research Gap and Objectives}


% 
% \subsubsection{Physics} 
% \label{section:physics}
% % You say polar waters are “strongly saturated with CO₂” — it’s true they hold high DIC, but “saturated” is misleading in the carbonate chemistry context.
% % You didn’t mention why physical processes matter for carbon uptake (ventilation, residence time, stratification) — that’s the key connection to the carbon cycle.
% % Heat/freshwater fluxes: the explanation is vague (e.g. “western boundary currents move warm water poleward” is true, but how does that connect to air–sea heat exchange and CO₂ solubility?).
% 
% The physical ocean carbon cycle is the part driven by physical processes. Due to advection, diffusion and stirring, the biochemical tracers are moved and spread in the global oceans.
% The Atlantic Meridional Overturning Circulation (AMOC) is a thermohaline circulation and one of the biggest planet-scale movements in the ocean. With the AMOC, Carbon, Nutrients and Heat get transported all over the globe. In the polar regions, typically, the Oceans' Deep water is formed. High saline and cold water, strongly saturated with Carbon dioxide, sinks in the ocean basin and flows equatorward. With this cold water movement, a balancing flow of warm tropical surface waters towards the pole is established. This balancing flow transports Plankton and Heat Poleward. With this transport of Heat, Nutrients and Carbon, the AMOC affects the ability of the surface ocean to take up additional CO2 and also impacts the long-term storage capabilities of the ocean.
% 
% % Upwelling/downwelling section: the description of Ekman transport and cyclone-driven divergence is garbled. Terms like “mass conversation” (should be conservation) are distracting errors, and the reasoning is not rigorous enough for a methods/theory section
% Ocean upwelling is the process of upward movement of deep water due to wind stress on surface waters. Ekman currents typically develop in coastal regions near land or when there are circular winds, like in a Cyclone. When the wind blows equatorward and parallel to the coastline, it induces a Coriolis-driven flow in the water perpendicular to the direction of the wind blowing. If this current moves surface waters away from the coastline, it induces deeper water upwelling due to mass conservation. If the surface current travels away from the coast, a decrease in surface water is induced. 
% In Cyclones, it follows a similar reasoning: if the surface current is induced away from the eye of the cyclone, upwelling is induced; otherwise, it will induce local downwelling.
% 
% % The air–sea flux section (momentum, heat, freshwater) reads like a loose afterthought, not connected back to the carbon cycle.
% These Ocean circulation patterns are mainly driven by the Air-Sea interaction and exchanges of Momentum, Heat, and Freshwater. Momentum transfer follows the general atmospheric circulation patterns and local winds. The heat transfer has the pattern of maximal heating in the tropics and cooling in the polar regions; near the western coastal area, there is usually higher cooling due to the western boundary currents moving warm water poleward. Freshwater exchange follows a complex pattern, mainly depending on regions of air rising in the atmosphere, typically near the tropics or mid-latitudes. Evaporation occurs in regions of warmer waters.
% 
% \subsubsection{Biology}
% Biology: biological pump basics (organic export, remineralization, role in CO2 uptake).
% 
% The atmospheric carbon dioxide enters the ocean through interactions at the sea surface or via coastal and riverine degradation fluxes \ref{section:physics}.
% Upon contact with the ocean water, the carbon dioxide undergoes a chemical reaction, resulting in the formation of carbonate and bicarbonate ions \ref{section:chemistry}.
% The next stage of the process involves ocean biology, which is responsible for the absorption of carbon within the biomass. 
% The resultant carbon, having become fixed, is exported into the deep ocean and subsequently redistributed through various processes.
% In this section, the focus will be on an analysis of the processes involved.
% 
% \begin{figure}
%     \centering
%     \includegraphics[width=0.75\linewidth]{figures/PP_dist_1998-2013.jpg}
%     \caption{The distribution of the mean net annual Primary Productivity throughout the global ocean based on satellite ocean colour data from 1998 through 2013 (LME boundaries are shown as black polygons) \citet{oreillyOReilly2017StatusTrends2017}}
%     \label{fig:PP_dist}
% \end{figure}
% 
% The primary biological activity in the ocean is that of phytoplankton and zooplankton. 
% The dissolved carbon is absorbed by phytoplankton in the primary production of the ecosystem. 
% The transformation of dissolved carbon into biomass is achieved through the process of photosynthesis, in which plankton play a pivotal role.
% During this process, they absorb carbon and nutrients (P, N, Fe) and concurrently release oxygen.
% The subsequent interaction between the phytoplankton and the zooplankton is of particular interest. The former is either grazed by the latter or dies naturally.
% Zooplankton serve as a source of nutrition for higher trophic levels in the ecosystem. They also export organic matter, either in the form of faecal pellets or as dead detritus, when they die.
% In addition to the fundamental biological cycle, the exported organic matter (detritus and faecal pellets) will be subject to the influence of bacteria as it descends and sinks into the deep ocean.
% The process of bacterial action involves the transformation of organic matter, resulting in the remineralisation of nutrients and the uptake of oxygen from the surrounding water.
% The organic matter, which has not undergone remineralisation, reaches the ocean floor, where the carbon is stored in the ocean's sediment.
% 
% In the majority of cases, the rates of photosynthesis and primary production are constrained by the availability of light and nutrients. Consequently, the greatest productivity is typically observed in the euphotic zone.
% Nutrients are typically supplied by upwelling \ref{section:physics}, vertical mixing, or riverine fluxes.
% The primary biolimiting nutrients are iron, silicate, nitrate, and phosphorus.
% Redfield Ratio
% 
% In the oceanic environment, a diverse array of phytoplankton species is present, each playing a distinct role in the carbon cycle.
% Diatoms are unicellular phytoplankton that form silica shells; this makes them significant in terms of the uptake and export of silicate to the deep ocean.
% 
% Coccolithophores are defined as autotrophic phytoplankton that also build cells, but in this case out of calcium carbonate (\ce{CaCO3}). Coccolithophores play a pivotal role in the binding of carbon, both within their biomass and in the composition of their shells. The coccolithophore shell is composed of calcium carbonate, a chemical compound that is found in the skeletons of many organisms. This chemical composition renders the species particularly sensitive to alterations in the carbonate saturation of the ocean's water. 
% In conditions of undersaturation, these organisms are unable to construct protective shells and may even experience adverse effects from the dissolution of their own shells, which can ultimately result in the demise of the species.
% 
% In addition to these shell-building plankton species, the ocean also hosts a significant number of heterotrophs, i.e. non-shell-building but phytosynthetic organisms.
% This comprehensive approach encompasses the three primary classifications of phytoplankton and encompasses all species depicted within the model.
% 
% \vspace{1cm}
%  
% Part two: Export and Remineralisation
% Maybe not that important for my topic,
% keep it short
% 
% While the primary production creates organic compounds out of the dissolved carbon in the ocean with photosynthesis, another big part of the biological carbon cycle is the export production of bacteria in the aphotic zone.
% Between 4-15\% of the primary production is exported to the deeper ocean and sinks through the ocean while being subject to bacterial activity.
% Most of the Organic matter is converted back to its inorganic constituents through remineralisation.
% Particulate matter sinking through gravity, as well as dissolved organic matter, moved with the circulation to the lower ocean, is affected by this process.
% In the Remineralisation process, bacteria convert organic matter and an oxidant to the inorganic Nutrients and additionally carbon dioxide and water. 
% The oxidant can either be dissolved oxygen in aerobic waters (respiration) or Nitrate (\ce{NO3}) in anaerobic water, which is known as denitrification. Then, in anaerobic conditions where nitrate is unavailable, Sulfate (\ce{SO4}) can be used, leading to sulfate reduction.
% 
% Due to this interaction with the nutrients and the liberation of the inorganic compounds, remineralisation plays a big role in the carbon and nutrient cycles.


% | Area          | Feedback  |
% | ------------- | -------------------------------------------------------------------------------------- |
% | **Structure** | Break into 3 subparagraphs (uptake, export/remineralisation, nutrient limits).         |
% | **Relevance** | Add 1–2 sentences linking biology → CO₂ balance → Omega / acidification.                   |
% | **Precision** | Remove physical/chemical overlaps; tighten terms (fixation, export, remineralisation). |
% | **Style**     | Simplify phrasing, avoid filler and repetition.                                        |
% | **Details**   | Integrate Redfield Ratio properly; clarify sediment burial fraction.                   |

% \subsubsection{Chemistry}
% \label{section:chemistry}
% More than 99\% of the DIC is stored in the form of bicarbonate and carbonate ions \citet{Williams2011}.
% 
% carbonate system basics (pH, DIC, TA, omega).
% In itself the Seawater is a special case of a system compared to the freshwater on land.
% Seawater has salts in it which makes the largest difference chemically and has major influence on the carbon cycle.
% Due to the dissolved salts the seawater is denser than regular freshwater, this density difference is together with density differences due to temperature the driver of the AMOC.
% 

% \subsection{Redfield Ratio}
% Fixed molar ratio of main chemical elements in living Cells.
% If we consider the Photosynthesis in living Phytoplankton, we get:
% \begin{equation}
%    \ce{106CO2 + 16 NO3- + H2PO4- + 122 H2O ( + Photons) -> C106H246O110N16P + 138 O2}
% \end{equation}
% From this endproduct, we can see the relation of $ C:N:P:\ce{O2} = 106:16:1:-138$, which is the so-called Redfield Ratio, which was measured by Redfield for C:N:P to estimate \ce{O2}.
% Redfield et al measured it for Zooplankton and Phytoplankton separately, but the now known Redfield Ratio is the mean of both measurements \citet{redfield1963}.
% \subsection{Liebigs Law}
% I don't remember what that is.
% 
% \subsection{Michaelis and Menten (Monod) Kinematics}
% Predator Prey Dynamics
% \subsection{Nutrient Recycling}
% 
% \begin{align*}
%     \frac{dN}{dt} &= -r_{max} \frac{NP}{k_N + N} +l_{PN} N + l_{DN} D\\
%     \frac{dP}{dt} &= r_{max} \frac{NP}{k_N + N} -l_{PN} P - l_{PD} P\\
%     \frac{dD}{dt} &= l_{PD} P - l_{DN} D\\
% \end{align*}
% 
% \subsection{Revelle factor}
% The Revelle factor (buffer factor) is the ratio of the change of Carbon dioxide to the change of dissolved inorganic carbon.
%   
% \begin{equation}
%   \text{Revelle factor} = \frac{\frac{\Delta \left[\ce{CO2}\right]}{\left[\ce{CO2}\right]}}{\frac{\Delta \left[DIC\right]}{\left[DIC\right]}}
% \end{equation}
% The Revelle factor is called a buffer factor since it defines how the ocean buffers the actual uptake of \ce{CO2} by converting it into Bicarbonates and Carbonate ions resulting in a smaller change of dissolved carbon dioxide compared to the actual uptaken carbon dioxide.
% The lower value the revelle factor has the larger is the buffering capacity of the ocean and more \ce{CO2} can be taken up from the atmosphere.

